% !TEX TS-program = xelatex
% !TEX encoding = UTF-8 Unicode
% !Mode:: "TeX:UTF-8"

%\documentclass[bachelor]{dutthesis} % 本科
\documentclass[master]{dutthesis} % 硕士
%\documentclass[doctor]{dutthesis} % 博士
\usepackage{amsmath}
\usepackage{amsfonts} 
\usepackage{bm} 
\usepackage{bbm} % ++++
\usepackage{algorithm}
% \usepackage{algorithmic}
\usepackage{algorithmicx}
\usepackage{algpseudocode}
\usepackage{subfigure}
\usepackage{tabularx}
\usepackage{arydshln}
\usepackage{array}
\usepackage{physics}
\usepackage{enumerate,paralist} %产生列表,有序,缩进等功能
\usepackage{layouts} %用于绘制页面布局
\usepackage{layout} %用于绘制页面布局
\usepackage{listings} %插入代码用,不用可以取消

\usepackage{multirow}

\lstset{frame=tlrb,fontadjust=true,basicstyle=\fontsize{10}{9}\ttfamily,
emphstyle=\color{red!80},
emph={root,base},emphstyle={\color{blue}},
tab=\rightarrowfill,keywordstyle={[1]\color{blue!80}},
keywordstyle={[2]\color{red!50}},
stringstyle=\color{magenta},
breaklines=true,
columns=fixed,
commentstyle=\color{black!50}}
 

%%%%%%%%%%%%%%%%%%%%%%%%%
%  使得layout使用mm做单位,不重要  %
%%%%%%%%%%%%%%%%%%%%%%%%%
\makeatletter
\renewcommand*{\lay@value}[2]{%
  \strip@pt\dimexpr0.351459\dimexpr\csname#2\endcsname\relax\relax mm%
}
\makeatother

% 修改算法中描述为中文
\floatname{algorithm}{算法}
\renewcommand{\algorithmicrequire}{\textbf{输入:}}
\renewcommand{\algorithmicensure}{\textbf{输出:}}
\renewcommand{\algorithmicfor}{循环}
\renewcommand{\algorithmicdo}{执行}
\renewcommand{\algorithmicend}{结束}
\renewcommand{\algorithmicif}{条件}
\renewcommand{\algorithmicthen}{则执行}

% 取消子图的自动编号:需要自行在标题中打出编号
\makeatletter
\renewcommand{\@thesubfigure}{\hskip\subfiglabelskip}
\makeatother

 % 这里是导言区
\begin{document}
\categorynumber{000} % 分类采用《中国图书资料分类法》
\UDC{000}            %《国际十进分类法UDC》的类号
\secretlevel{公开}    %学位论文密级分为"公开"、"内部"、"秘密"和"机密"四种
\studentid{21909145}   %学号要完整,前面的零不能省略。

\title{大连理工大学人体感知实验室\TeX 模板}{}{The \TeX\ Template for DUT-HPC Group}{}
%%%%%%%%%%%%%%%%%%%%%%%%%%%%%%%%%%%%%%%%%%%%%%%%%%%%%%%%%%%%%%
%  title包含四个定义,分别是中文题目,中文副标题,
%英文标题,以及英文副标题,没有副标题可以空{}不能不写
%
%%%%%%%%%%%%%%%%%%%%%%%%%%%%%%%%%%%%%%%%%%%%%%%%%%%%%%%%%%%

\author{DUT-HPC}{DUT-HPC}
\advisor{DUT-HPC}{课题组}{DUT-HPC}{Group}
\degree{工学硕士} % 详细学位名称
\major[12em]{计算机科学与技术}
\defenddate{2022.4.28}
\maketitle

\begin{abstract}{人体运动;感知}

这是中文摘要部分。

\end{abstract}

\begin{englishabstract}{Human Action; Perception}

This is the English Abstract.
	
\end{englishabstract}

\let\cleardoublepage\clearpage

\tableofcontents % 中文目录
% \tableofengcontents % 英文目录

\cleardoublepage

% \tableoffigurecontents % 图目录
% \tableoftablecontents % 表目录

%\listoffigures
%\listoftables
\stcleardp

% \input{SMTable.tex} % 论文数学符号定义页

\begin{Main} % 开始正文

\chapter{第一章}

测试缩进测试缩进测试缩进测试缩进测试缩进测试缩进测试缩进测试缩进测试缩进测试缩进测试缩进测试缩进测试缩进测试缩进测试缩进测试缩进测试缩进

\section{公式的书写}

\begin{equation}
\label{eq-1}
    \mathbf{A} = \mathbf{B} \mathbf{C}^T
\end{equation}

\section{图的绘制}

\begin{figure}
    \vspace{1em}
    \centering
    \includegraphics[width=0.7\linewidth]{figures/master-hwzs.pdf}
    \bicaption{测试图\;\!ying\;\!宋}{Example}
    \label{fig:example}
\end{figure}

引用\autoref{fig:example}

\section{三线表的绘制}

引用\autoref{tab:example}

\begin{table}[h]
    \bicaption{三线表示例\;\!ying\;\!宋}{Table example}
    \label{tab:example}
    \centering
    \vspace{-0.2cm}
    \wuhao\yingsong
    \begin{tabular}{cc}
        \hline
        列1 & 列2  \\
        \hline
        数据1 & 数据2 \\
        数据3 & 数据4 \\
        \hline
    \end{tabular}
\end{table}

\section{算法流程的书写}

\begin{figure}
    \centering
    \begin{minipage}{0.75\linewidth}
        \begin{algorithm}[H]
            \caption{示例算法}
            \label{alg-1}
            \begin{algorithmic}[1]
                \Require 原始数据 $\mathbf{X}$
                \Ensure 结果 $\mathbf{Y}$
                \State 普通语句;
                \For{$i = 1 : n$}
                    \State 循环逻辑;
                \EndFor
            \end{algorithmic}
        \end{algorithm}
    \end{minipage}
\end{figure}

\section{列表的使用}

\begin{asparaenum}[(1)]
    \item 第一项;测试缩进测试缩进测试缩进测试缩进测试缩进测试缩进测试缩进测试缩进测试缩进测试缩进测试缩进测试缩进测试缩进测试缩进测试缩进测试缩进测试缩进
    \item 第二项。
\end{asparaenum}

\section{测试首行缩进}

测试缩进测试缩进测试缩进测试缩进测试缩进测试缩进测试缩进测试缩进测试缩进测试缩进测试缩进测试缩进测试缩进测试缩进测试缩进测试缩进测试


% 清除空白页
\let\cleardoublepage\clearpage




\chapter{第二章}

\section{这是节}

\subsection{这是小节}

% 不要有四级标题,即\subsubsection{xx}

所有的引用可以使用autoref命令,会自动匹配公式、图、表、章节等。例如,\autoref{eq-1}。

引用文献\cite{trozzi2021umap,wang2012entropy}。


% 清除空白页
\let\cleardoublepage\clearpage





% 结论
\begin{Conclusion}

这是结论部分。

\end{Conclusion}

\end{Main} % 结束正文

\begin{Reference}
    \yingsong\bibliography{bibtex.bib}
\end{Reference}

\let\cleardoublepage\clearpage

\begin{Publics}
  \begin{enumerate}[1.]
    \item 论文名. 作者. 期刊名, 年份, 卷号: 页码. 主办单位:Elsevier。 SCI检索期刊,SCI检索号:121212121121(本硕士学位论文第x章)
  \end{enumerate}
\end{Publics}

\begin{Acknowledgement}

  这是致谢部分。

\end{Acknowledgement}

\newpage
\printindex % 索引
\authorizationpage

\end{document}